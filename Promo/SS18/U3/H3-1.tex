\documentclass[10pt,a4paper]{article}
\usepackage[utf8]{inputenc}
\usepackage{amsmath}
\usepackage{amsfonts}
\usepackage{amssymb}
\usepackage{graphicx}
\usepackage{listings}
\author{Andrea Colarieti}
\title{Programmierung und Modellierung U3 H3-1}


\begin{document}
\section{Übungsblatt 3 Lösung H3-1}
Andrea Colarieti Tosti
\newline\newline


\begin{lstlisting}
\end{lstlisting}

Die behandelte Funktion:
\begin{lstlisting}
magic :: Integer -> Integer -> Integer -> String
magic a b c | even c, b > 0, a < 0 = 'x' : (magic (1+a) b (1+c))
            | odd c, b > 0, a < 0 = 'y' : (magic a (b-1) (c+1))
            | otherwise = show c

\end{lstlisting}

\subsection{A}
\begin{lstlisting}
max a+c 0
\end{lstlisting} 
ist keine gute abstiegsfunktion. wir zeigen es anhand von dem fall

magic (-3) 2 5 = 8
max (-3+5) 0 = 2 $ \ngtr 8$
\newline
\subsection{B}
Die Funktion $ \int(x,y,z) $ hat ein Domain $ A = dom(f)$
$ A = \mathbb{Z}\times\mathbb{Z}\times\mathbb{Z} \mapsto \mathbb{Z}$
\newline
$odd$ und $even$ decken alle fälle ab zusätzlich wird geprüft, ob $b > 0$ oder $a < 0$.
Sollte $a>0$ oder $b<0$ sein, wird $c$ sofort zurückgegeben.
Im Fall dass alle Bedingungen true ergeben, wird, bei einem geraden Wert von $c$, $a$ 
hochgezählt oder im anderen Fall wird $b$ runtergezählt.
Dabei wächst $c$ immer um 1.
Als Abstiegsfunktion würde ich  max $((0-x)(0+y)+z)$ 0 nehmen.
\newline
Für den fall $f(x,y,z) | x = -3 , y = 2 , z = 5 \Rightarrow f(x,y,z)= 8 $
Dagegen ist $ max ((0-3)(0+2)+5) = 11 > 8 $

\end{document}