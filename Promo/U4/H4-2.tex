\documentclass[10pt,a4paper]{article}
\usepackage[utf8]{inputenc}
\usepackage{amsmath}
\usepackage{amsfonts}
\usepackage{amssymb}
\usepackage{graphicx}
\usepackage{listings}
\author{Andrea Colarieti}
\title{ProMo Übung 4 H4-2 Lösung}


\begin{document}
\maketitle

H4-2 Induktion mit Listen II
(2 Punkte; Abgabe: H4-2.txt oder H4-2.pdf)
Es sei vs und ws zwei beliebige Listen, weiterhin sei $n = |vs|$. Beweisen
Sie mit Induktion über die Länge $n \in \mathbb{N}$ von vs, dass zip vs ws = min(vs, ws) gilt.
\newline
\begin{lstlisting}
zip :: [a] -> [b] -> [(a,b)]
zip 	[] 	 _   = []
zip 	_  	 []  = []
zip (x:xs) (y:ys)  = (x,y) : zip xs ys
\end{lstlisting}

\section{Lösung}
Seien vs und ws 2 Listen mit Länge n und m, $n \in \mathbb{N}$ und $m \in \mathbb{N}$.
\newline
Ind. Annahme: 
Das Verhalten der Funktion zip ändert sich nicht beim Wachsen der länge der Liste im ersten parameter. \newline Also gilt zip vs ws = min (n+1,m)  $\forall n+1 \in \mathbb{N} $ 1mit $ n + 1 \leq m$
\newline\newline
Sei n = 0 $\Rightarrow$
\newline 
zip [ ] ws = [ ]\newline
$\Leftrightarrow$min(0,m) = 0 
\newline\newline
Ind Anfang: $n = 1 $\newline
zip a:[ ] (b:ws) $\Rightarrow$\newline (a,b): zip [ ] (c:ws) $\Rightarrow$\newline (a,b):[ ] = (a,b)\newline
$\Leftrightarrow$min(1, m) + min(0,m) = 1 + 0
\newline\newline
Ind. Schritt: $n \mapsto n + 1 $\newline 
zip (a:vs) (b:ws) $\Rightarrow$\newline (a,b):zip (c:vs) (d:ws) $\Rightarrow$\newline (a,b):(c,d):zip (e:vs) (f:ws) = (a,b):(c,d):...:zip [ ] (f:ws) = \newline (a,b):(c,d):(e,f):.....:[ ]\newline
$\Leftrightarrow$min(n, m) + min(1,m) + min(0,m) = n + 1 + 0 = n + 1 \newline
\begin{flushright}
\hfill $\square$
\end{flushright}
\end{document}