\documentclass[12pt,a4paper]{article}
\usepackage[utf8]{inputenc}
\usepackage{amsmath}
\usepackage{amsfonts}
\usepackage{amssymb}
\author{Andrea Colarieti Tosti}
\title{Rechnernetze und Verteilte Systeme Übungsblatt 00}

\begin{document}
\maketitle
\section*{Übung 1}
\subsection*{a)}
0 1 2 3 4 5 6 7 8 9 A B C D E F 
\subsection*{b)}
\begin{center}
 \begin{tabular}{|c | c | c | c |} 
 \hline
 Dezimal & Binär & Oktal & Hex \\ [0.5ex] 
 \hline
 2 & 10 & 2 & 2 \\ 
 \hline
 4 & 100 & 4 & 4 \\
 \hline
 8 & 1000 & 10 & 8 \\
 \hline
 10 & 1010 & 12 & A \\
 \hline
\end{tabular}
\end{center}
\subsection*{c)}
\begin{center}
 \begin{tabular}{|c | c | c | } 
 \hline
 Dezimal & Binär & Hex \\ [0.5ex] 
 \hline
 16 & 10000 & 10 \\ 
 \hline
 127 & 1111111 & 7F \\
 \hline
 168 & 10101000 & A8 \\
 \hline
 172 & 10101100 & AC \\
 \hline
 192 & 11000000 & C0 \\
 \hline
 255 &  11111111 & FF \\
 \hline
\end{tabular}
\end{center}\newpage
\subsection*{d)}
Die Zahl hat 32 Stellen \\
32*1\\0*1
\section*{Übung 2}
\subsection*{a)}
Beide Bäume haben jeweils 4 Blätter.
\subsection*{b)}
Erster Pfad: 1 1 0 0 \\
Zweiter Pfad: 1 0 0
\subsection*{c)}
$ 2^n - 1$
\subsection*{d)}
Ja, ein pfad ist immer eindeutig und zeigt immer auf ein Knoten oder ein Blatt.
\subsection*{e)}
0 0 0 0 0 0 0
\subsection*{g)}
Sie haben den gleichen Pfad von der Wurzel bis zu ihrem Vorfahren.
\subsection*{h)}
\paragraph{i} 49320
\paragraph{ii} C0A8
\paragraph{iii} 192 168
\section*{Übung 3}
\subsection*{a)}
7+6+5+4+3+2+1 = 28
\subsection*{b)}
$ n(n-1)/2 $

\end{document}