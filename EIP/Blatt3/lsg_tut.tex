\documentclass[12pt,a4paper]{article}
\usepackage[utf8]{inputenc}
\usepackage{amsmath}
\usepackage{amsfonts}
\usepackage{amssymb}
\usepackage{makeidx}
\usepackage{graphicx}
\author{Andrea Colarieti Tosti}
\title{EIP B3 lsg tutorium}

\begin{document}
\maketitle
\section{3.1}
\subsection{Aufg 1}
\subsubsection{a)}
$ n^2 > 2n+1 \Rightarrow 3 \leq n \in \mathbb{N}$\\
IA $ n = 3  \Rightarrow 3^2=9 > 7 = 2 \cdot 3 +1 $ \\
Es gibt ein $n \geq $, sodass $n^2 \geq 2n+1$ gilt. \\
Beachte $n+1$ \\
$(n+1)^2 = n^2+n+1 > (2n+1) +2n+1 > 2n+1 +1+1 = 2n+2+1 = 2 \cdot (n+1)+1$
\subsubsection{b)}
$2n> n^2 $ für $5 \geq n \in \mathbb{N}$ \\
IA: $ n = 5 : 2^5 = 32 > 25 = 5^2 $\\
IV:  es gibt ein $ n \geq5 $  sodass $ 2^n > n^2$ gibt\\
IS: Beachte $n+1$:\\
$ 2^{n+1} = 2 \cdot 2^n > 2\cdot n^2 = n^2 + n^2 > n^2+2n+1 = (n+1)^2$\\
\subsubsection{c)}
$ x \in \mathbb{R}, x \geq -1, n\in \mathbb{N}$\\
IA: $n=1 : (1+x)^1= 1+x = 1+1x$\\
IV: Es gibt ein n, sodass $ (x+1)^n \geq 1+nx$ gilt.\\
IS: $ (1+x)^{n+1} = (x+1)\cdot (x+1)^n \geq (1+x)(1+nx) = 1+x+nx+nx^2 = 1+x+nx+0 = 1+(n+1)x$
\end{document}