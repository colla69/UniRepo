\documentclass[12pt,a4paper]{article}
\usepackage[utf8]{inputenc}
\usepackage{amsmath}
\usepackage{amsfonts}
\usepackage{amssymb}
\usepackage{graphicx}
\usepackage{imakeidx}
\author{Andrea Colarieti Tosti}
\title{Einführung in die Programmierung Blatt 2 Lösung}

\makeindex

\begin{document}
\maketitle

\section*{Aufgabe 1}
\subsection*{a)}
$$
  a \vee b = \neg ( \neg a \wedge \neg b)
$$
\subsection*{b)}
NAND ist wie folgt definiert :
$  a \uparrow b = \neg (a \wedge b) $\\
Wir versuchen die junktoren $\{ \neg ,\wedge,\vee\}$ mit $\uparrow$ darzustellen: \\
$$
\neg \Rightarrow \neg a = a \uparrow a
$$ $$
\wedge \Rightarrow a \wedge b = ((a \uparrow b) \uparrow (a \uparrow b))
$$ $$
\vee \Rightarrow a \vee b = ((a \uparrow a) \uparrow (b \uparrow b)) 
$$
\subsection*{c)}
mächtigkeit aus der multiplication der möglichen ergebnisse.\\ auf 2 inputs gibt es 4 mögliche ausfühungen (wahrheitstabelle) diese hat 4 outputs die T oder F sind ..\\
also berechnen wir $2^4$\\
$(2^n)^2 = 2^{2n}$ also für 3 folgt $(2^{2\cdot 3}) = 2^6 = 64$
\section*{Aufgabe 2}
\subsection*{a)}
Zu zeigen ist das folgende gilt: $ 1+2+4+8+...+2^n = 2^{n+1} -1$\\
Induktionsanfang: für n = 0 \\
$ 1 = 2^{0+1}-1=1$\\
Induktionsannahme: für n gilt:
$\sum_{k=o}^n 2^n = 2^{n+1}-1 $\\
Induktionsschritt:\\
$\sum_{k=o}^n 2^n + 2^{n+1} = 2^{n+1}-1 + 2^{n+1} = 2 \cdot 2^{n+1}-1 = 2^{n+2}-1 $
\begin{flushright}
q.e.d.
\end{flushright}
\subsection*{b)}
Induktionsanfang: \\
$ a_2 = 2- \frac{1}{\frac{n+1}{n}} = 2-\frac{n}{n+1} = \frac{2n+2-n}{n+1} = \frac{n+2}{n+1}$\\\\
Induktionsschluss:\\
$ a_{n+1} = 2 - \frac{1}{\frac{n+1}{n}} = 2 - \frac{n}{n+1} = \frac{2n+2-n}{n+1} = \frac{n+2}{n+1}$
\begin{flushright} q.e.d. \end{flushright}
\section*{Aufgabe 3}
\subsection*{a)}
\textit{Die Funktion evaluate(t) ist Injektiv da der Baum immer mehr breiter wird.}
nooope die ist doch surjektiv aus dem folgenden Grund: \\
evaluate(t) gibt true oder false zurück aber die menge der möglichen terme unendlich ist.. me so mongo!
\subsection*{b)}
$$
 h( X \in \sum) $$ $$
 h( \neg (t)) = 1 + h(t) $$ $$
 h(\wedge (t_1,t_2)) = 1 + max( h(t_1),h(t_2) )$$ $$
 h(\vee (t_1,t_2)) = 1 + max( h(t_1),h(t_2) )$$ $$
$$
\subsection*{c)}
NNF ( $ X \in \sum$)

\end{document}