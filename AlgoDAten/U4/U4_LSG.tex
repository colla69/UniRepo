\documentclass[12pt,a4paper]{article}
\usepackage[utf8]{inputenc}
\usepackage{amsmath}
\usepackage{amsfonts}
\usepackage{amssymb}
\usepackage{graphicx}
\usepackage{lmodern}
\usepackage{kpfonts}
\author{Andrea Colarieti Tosti}
\title{Algorithmen und Datenstrukturen Übungsblatt 4 Lösung}
\begin{document}
\maketitle
\section{Aufgabe Global 4-1}
Um die drei schnellsten Pferde zu bestimmen sind 10 Rennen Nötig.\\
$12=5+3+2+1+1$ \\
Wir müssen alle Pferde miteinander vergleichen also lassen wir alle Pferde an 5 Rennen  teilnehmen. Aus den Ergebnisse nehmen wir jeweils das Podium raus und bilden somit eine neue gruppe, die aus 5*3 Pferde besteht.\\
Diesmal brauchen wir 3 Rennen um alle Pferde zu vergleichen. Wir nehmen wieder die 3 schnellten aus jeder Gruppe raus und haben noch 9 Pferde zum vergleichen.\\
Wir organisieren 2 weitere Rennen einmal mit 5 und einmal mit 4 Teilnehmer.
Aus den 6 schnellsten lassen wir nochmal ein Rennen mit 5 Teilnehmer stattfinden und wir lassen die 3 schnellsten gegen den letzteren Pferd rennen und finden die 3 schnellsten Pferde aus der gruppe heraus.\\
Die nummer 3 wurde nicht willkürlich ausgesucht, es ist die kleinste Menge aus einer Gruppe in der die 3 schnellsten Pferde aus 25 enthalten sein konnten.

\section{Aufgabe Global 4-2}

\end{document}